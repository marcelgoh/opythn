\documentclass[11pt, twoside]{article}
\usepackage[margin=1in]{geometry}
\usepackage{fancyhdr, listings, color, titling, inconsolata}
\pagestyle{fancy}
\usepackage[shortlabels]{enumitem}
\setlist{  
  listparindent=\parindent,
  parsep=0pt,
}

\lstset{
    basicstyle=\ttfamily,
    backgroundcolor=\color[rgb]{0.9,0.9,0.9},
    keywordstyle=\color[rgb]{0,0,1},
    commentstyle=\color[rgb]{0.8, 0.4, 0.2},
    stringstyle=\color[rgb]{0.45,0.345,0.484},
}

\setlength{\droptitle}{5em}
\setlength{\headheight}{14pt}

\newcommand{\ms}{\texttt}

\fancyhead[RE]{NPRG045 Project Specification: OPythn}
\fancyhead[LE]{\thepage}
\fancyhead[LO]{Marcel Goh}
\fancyhead[RO]{\thepage}
\cfoot{}



\begin{document}
\title{\Huge{\textbf{NPRG045 Project Specification: OPythn}}}
\author{\Large{Marcel Goh}}
\clearpage\maketitle
\thispagestyle{empty}
\newpage
\setcounter{page}{1}

\section{Target}
    The OPythn project aims to implement a working subset of the Python programming language by means of a bytecode compiler and interpreter, written in OCaml. Users will be able to interact with OPythn via a top-level read-eval-print loop or compile OPythn source code to bytecode. OPythn will run on Unix operating systems, including macOS and Ubuntu.

\section{Language}
    The core of OPythn is designed to be lightweight and minimal. Basic types and operations, control structures, and elementary data structures are included, while more complex Python constructions, such as anonymous functions, list comprehensions, generators, and coroutines are omitted.
    \subsection{Features}
    OPythn inherits the following features directly from Python:
    \begin{itemize}
        \item Primitive types \ms{int}, \ms{float}, \ms{str}, and \ms{bool}
        \item Arithmetic and boolean operators
        \item Control structures \ms{if}, \ms{for}, and \ms{while}
        \item Lists and dictionaries
        \item Named functions
        \item Classes and objects
    \end{itemize}
    To round out OPythn's standard library, other core functions will be defined in OPythn and included in the standard OPythn installation. These mainly consist of basic operations on mathematical objects, lists, and strings.
    \subsection{Lexical Conventions}
    The following are OPythn keywords and cannot be used in ordinary names:
    \begin{center}
        \begin{tabular}{cccccc}
            \ms{int} & \ms{float} & \ms{str} & \ms{bool} & \ms{def} & \ms{return}\\
            \ms{True} & \ms{False} & \ms{None} & \ms{and} & \ms{or} & \ms{not}\\
            \ms{if} & \ms{elif} & \ms{else} & \ms{for} & \ms{in} & \ms{while}\\
            \ms{break} & \ms{continue} & \ms{class} & \ms{is} & \ms{del} & \ms{import}\\
            \ms{in} & \ms{from} & \ms{as} &&&
    \end{tabular}
    \end{center}
    The rules that the lexer uses to recognise identifiers, numbers, and special characters are defined by the following regular expressions:
    \begin{lstlisting}[language=python]
    ENDMARKER: '\Z'
    NAME: '[^\d\W]\w*'
    NEWLINE: '\n'
    NUMBER: '[+-]?((\d+\.\d*|\d*\.\d+)'
    STRING: '(\"[^\n\"\\]\")|(\'[^\n\'\\]\')'
    \end{lstlisting}

    \subsection{Grammar}
    This is the complete OPythn grammar specification:
    \begin{lstlisting}[language=python]
    # Start symbols
    single_input: NEWLINE | simple_stmt | compound_stmt NEWLINE
    file_input: (NEWLINE | stmt)* ENDMARKER
    
    funcdef: 'def' NAME parameters ':' suite
    classdef: 'class' NAME ['(' [arglist] ')'] ':' suite
    parameters: '(' [arglist] ')'
    vararglist: arg (',' arg)*
    arg: NAME
    
    stmt: simple_stmt | compound_stmt
    simple_stmt: small_stmt (';' small_stmt)* [';'] NEWLINE
    small_stmt: (expr_stmt | del_stmt | flow_stmt | import_stmt)
    expr_stmt:  
    compound_stmt: if_stmt | while_stmt | for_stmt | funcdef |
        classdef
    
    # Control structures
    if_stmt: 'if' test ':' suite ('elif' test ':' suite)*
        ['else' ':' suite]
    while_stmt: 'while' test ':' suite ['else' ':' suite]
    for_stmt: 'for' exprlist 'in' testlist ':' suite
        ['else' ':' suite]
    
    # Tests and expressions
    test: or_test ['if' or_test 'else' test]
    or_test: and_test ('or' and_test)*
    and_test: not_test ('and' not_test)*
    not_test: 'not' not_test | comparison
    comparison: expr (comp_op expr)*
    comp_op: '<'|'>'|'=='|'>='|'<='|'!='|'in'|'not' 'in'|
        'is'|'is' 'not'
    expr: xor_expr ('|' xor_expr)*
    xor_expr: and_expr ('^' and_expr)*
    and_expr: shift_expr ('&' shift_expr)*
    shift_expr: arith_expr (('<<'|'>>') arith_expr)*
    arith_expr: term (('+'|'-') term)*
    term: factor (('*'|'/'|'%'|'//') factor)*
    factor: ('+'|'-'|'~') factor | power
    power: atom_expr ['**' factor]
    atom_expr: atom trailer*
    atom: (NAME | NUMBER | STRING+ | '...' | 'None' |
        'True' | 'False')
    trailer: '(' [arglist] ')' | '[' subscript ']' | . NAME
    subscriptlist: subscript (',' subscript)* [',']
    subscript: test | [test] ':' [test] [sliceop]
    sliceop: ':' [test]
    exprlist: expr (',' expr)* [',']
    testlist: test (',' test)* [',']
    dictorsetmaker: ( ((test ':' test | '**' expr)
                       ((',' (test ':' test)* [','])) |
                      (test (',' test)* [','])) )
    arglist: argument (',' argument)* [',']

    argument (test | test '=' test)
    \end{lstlisting}


\section{Implementation}
Both the OPythn bytecode compiler and interpreter will be implemented in OCaml. This section gives a rough overview of the process by which OPythn source code is handled and executed.
    \subsection{Input/Output}
    The OPythn program can be run directly in a Unix terminal via the command \texttt{opythn}. When run without arguments, this command launches a read-eval-print loop into which the user can enter commands. In interactive mode, the interpreter evaluates a valid command as soon as it is received:
    \begin{lstlisting}
    OPythn (Interactive), version 0.0.0
    :? for help, :q to quit
    ]=> print("Hello, world!")
    Hello, world!
    \end{lstlisting}
    Alternatively, OPythn source code can be defined in a separate \texttt{.opy} file and provided to the interpreter as a program argument. In this, mode, OPythn immediately compiles and interprets the source code directly.
    \subsection{Lexical Analysis}
    Upon reading source code, the OPythn front-end passes the data as a string to a series of functions that lex the code into tokens. The lexer will be created with the help of \texttt{ocamllex}, a program that generates a deterministic finite automaton in OCaml. This resulting lexer matches regular expressions in the string to convert chunks of characters into the correct tokens. 
    \subsection{Parsing}
    The tokens will then be fed into a parser, which will be created according to the grammar rules outlined in Section 2.3 and using \texttt{ocamlyacc}. The result will be an abstract syntax tree that represents the structure and semantics of the OPythn program. 


\begin{thebibliography}{1}
    \bibitem{pythonref} Guido van Rossum. \textit{The Python Language Reference}. Python Software Foundation, 2019.
\end{thebibliography}

\end{document}
