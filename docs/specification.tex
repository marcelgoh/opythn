\documentclass[11pt, twoside]{article}
\usepackage[margin=1in]{geometry}
\usepackage[T1]{fontenc}
\usepackage{fancyhdr, listings, color, titling, inconsolata, textcomp, algorithm}
\usepackage{algpseudocode}
\pagestyle{fancy}
\usepackage[shortlabels]{enumitem}
\setlist{  
  listparindent=\parindent,
  parsep=0pt,
}

\lstset{
    basicstyle=\ttfamily,
    backgroundcolor=\color[rgb]{0.9,0.9,0.9},
    keywordstyle=\color[rgb]{0.97,0,0.369},
    commentstyle=\color[rgb]{0.8, 0.4, 0.2},
    stringstyle=\color[rgb]{0.45,0.345,0.484},
    showstringspaces=false,
    upquote=true
}


\setlength{\droptitle}{5em}
\setlength{\headheight}{14pt}

\newcommand{\ms}{\texttt}
\newcommand{\sq}{\textquotesingle}

\fancyhead[RE]{NPRG045 Project Specification: OPythn}
\fancyhead[LE]{\thepage}
\fancyhead[LO]{Marcel Goh}
\fancyhead[RO]{\thepage}
\cfoot{}



\begin{document}
\title{\Huge{\textbf{NPRG045 Project Specification: OPythn}}}
\author{\Large{Marcel Goh}}
\clearpage\maketitle
\thispagestyle{empty}
\newpage
\setcounter{page}{1}

% INTRODUCTION
\section{Introduction}
    The OPythn project aims to implement a working subset of the Python programming language by means of a bytecode compiler and interpreter, written in OCaml. Users will be able to interact with OPythn via a top-level read-eval-print loop or compile OPythn source code to bytecode. OPythn will run on Unix operating systems, including macOS and Ubuntu.

% LEXICAL CONVENTIONS
\section{Lexical Conventions}
    OPythn reads source code as ASCII characters, which are fed into a lexer. This section describes how the tokenises source code input.
    \subsection {Line and block structure}
    The end of a line is represented by the \ms{NEWLINE} token. In general, simple statements cannot be split over multiple lines. However, an expression in parentheses, square brackets, or curly braces can be written over multiple lines without ending the statement. The \ms{\#} character can be used to indicate comments in source code. Any letters from \ms{\#} to the end of the line will be ignored by the lexer. A line that contains only whitespaces or comments is ignored except during interactive evaluation, when a blank line is used to indicate the end of a multi-line statement.\\
    \indent In OPythn, the amount of leading whitespace at the beginning of a line indicates the indentation level of the line. A tab character counts for exactly four spaces during this computation. Indentation levels of consecutive lines are used to generate \ms{INDENT} and \ms{DEDENT} tokens. This is done by means of a stack using the following algorithm, as described by the standard Python reference \cite{pythonref}.
    \begin{algorithm}
    \caption{Insertion of \ms{INDENT} and \ms{DEDENT} tokens (as part of general lexing procedure)}
    \begin{algorithmic}[1]
    \Procedure{lex}{OPythn source code}
        \State $token\_list \gets$ create a new empty list
        \State $stack \gets$ create a new empty stack
        \State $stack.push(0)$
        \While{there are lines to be read}
            \State $line \gets$ the current line
            \State $num\_sp \gets$ the number of leading spaces in $line$
            \If{$curr > stack.peek()$}
                $stack.push(curr)$
                $token\_list.add(\ms{INDENT})$
            \ElsIf{$curr < stack.peek()$}
                \State $count\gets 0$
                \While{$curr < stack.peek()$}
                    \State $stack.pop()$
                    \State $count\gets count + 1$
                \EndWhile
                \For{$i\gets 0$, $count$}
                    \State $token\_list.add(\ms{DEDENT})$
                \EndFor
                \EndIf
                \State tokenise the rest of the line and add tokens to $token\_list$
                \algstore{indent}
    \end{algorithmic}
    \end{algorithm}
    \begin{algorithm}
    \begin{algorithmic}
            \algrestore{indent}
        \EndWhile
        \State \Return{$token\_list$}
    \EndProcedure
    \end{algorithmic}
    \end{algorithm}

    \subsection{Keywords and identifiers}
    The following are OPythn keywords and cannot be used in ordinary identifiers:
    \begin{center}
        \begin{tabular}{cccccc}
            \ms{int} & \ms{float} & \ms{str} & \ms{bool} & \ms{def} & \ms{return}\\
            \ms{True} & \ms{False} & \ms{None} & \ms{and} & \ms{or} & \ms{not}\\
            \ms{if} & \ms{elif} & \ms{else} & \ms{for} & \ms{in} & \ms{while}\\
            \ms{break} & \ms{continue} & \ms{class} & \ms{is} & \ms{del} & \ms{import}\\
            \ms{in} & \ms{from} & \ms{as} &\ms{global} & \ms{nonlocal} & \ms{lambda}
    \end{tabular}
    \end{center}
    Any string that is not a keyword and contains uppercase and lowercase letters, underscores, and (except for the first character) numbers is a valid identifier. Case is significant. Some names are reserved for use by the standard library, which is described in Section 3.
    \subsection{Literals}
    Strings are enclosed in matching single quotes or double quotes. Triple-quoted strings are not supported, nor are formatted string literals. The backslash character \ms{\textbackslash} is used to escape characters that normally have a special meaning.
    \begin{lstlisting}[language=python]
    string: "'" stringitem* "'" | '"' stringitem '"'
    stringitem: stringchar | stringescapeseq
    stringchar: <source character besides '\\', '\n', "'", '"'>
    stringescapeseq: '\\' <source character>
    \end{lstlisting}
    \indent There are two types of numeric literals in OPythn: integers and floating-point numbers. Leading zeroes in a non-zero decimal are not allowed. Underscores are also not allowed in numeric literals, unlike in Python 3.
    \begin{lstlisting}[language=python]
    integer: decinteger | bininteger | octinteger | hexinteger
    decinteger: nonzerodigit digit* | '0'+
    bininteger: '0' ('b'|'B') bindigit+
    octinteger: '0' ('o'|'O') octdigit+
    hexinteger: '0' ('x'|'X') hexdigit+
    nonzerodigit: '1'...'9'
    digit: '0'...'9'
    bindigit: '0' | '1'
    octdigit: '0'...'7'
    hexdigit: digit | 'a'...'f' | 'A'...'F'

    floatnumber: pointfloat | exponentfloat
    pointfloat: [digit+] fraction | digit+ '.'
    exponentfloat: (digit+ | pointfloat) exponent
    fraction: '.' digit+
    exponent: ('e'|'E') ['+' | '-'] digit+
    \end{lstlisting}
    \subsection{Operators and delimiters}
    All operators and delimiters are inherited from standard Python except the \ms{@} and \ms{@=} symbols. (They are used for matrix multiplication and decorators, both of which are not features of OPythn.)
    \subsection{Token representation}
    The lexer will produce tokens of the following OCaml datatype:
    \begin{lstlisting}[language=caml]
    type token =
      NEWLINE | INDENT | DEDENT
    | NAME of string  (* identifier *)
    | KWORD of string
    | NUMBER of string
    | STRING of string
    | OP of string
    | DELIM of string
    \end{lstlisting}

% LANGUAGE
\section{Language}
    The core of OPythn is designed to be lightweight and minimal. Basic types and operations, control structures, and elementary data structures are included, while more complex Python constructions, such as list comprehensions, generators, and coroutines are omitted.
    \subsection{Simple and compound types}
    Every value in OPythn is of a certain type. There are four primitive types (\ms{int}, \ms{float}, \ms{str}, and \ms{bool}) and all of them are immutable. Integers are represented with 63 bits and can range between -4,611,686,018,427,387,904 and 4,611,686,018,427,387,903 inclusive.\\
    \indent OPythn has three compound types: \ms{list}, \ms{dict}, and \ms{tuple}. Of these, only tuples are immutable. Lists are zero-indexed and accessing an element via its index takes constant time. Just like in Python, lists support negative indexing and slice operators. For example, if \ms{a = [1,True,3,'hi',5]}, then \ms{a[1:3]} would give the list \ms{[True,3]} and \ms{a[-2]} would return the element \ms{'hi'}. The syntax is similar for OPythn tuples, except that because tuples are immutable, the user cannot modify values (whereas this is possible with a list). Dictionaries support values of different types, with the restriction that keys must be of an immutable type. For example, \ms{d = \{2: 'bonjour', False: 3.3\}} is a valid OPythn dictionary. Objects of a user-defined class cannot be keys in a dictionary.\\
    \indent Typed values are represented as follows:
    \begin{lstlisting}[language=caml]
    type py_immut =
      INT of int
    | FLOAT of float
    | STR of string
    | BOOL of bool
    | TUPLE of py_val array
    | TYPE of string
    | FUN of ...
    | NONE
    \end{lstlisting}
    There will be a separate type for OPythn compound types and then a general \ms{py\_val} type to capture both of these possibilities. To represent an OPythn list, we will need to create a resizing array type in OCaml. To represent a dictionary, we use an OCaml hashtable.
    \begin{lstlisting}[language=caml]
    type py_mut =
      LIST of py_val Resizable.t (* dynamic array type *)
    | DICT of (py_immut, py_val) Hashtbl.t
    | OBJ of ...
    and type py_val =
      IMMUT of py_immut
    | MUT of py_mut
    \end{lstlisting}

    \subsection{Operators and control structures}
    Ordinary arithmetic operators for addition, multiplication, subtraction, division, modulus, and exponentiation are supported between types \ms{int} and \ms{float}. Additionally, OPythn supports the bitwise operations \ms{|}, \ms{\^}, \ms{\&}, \ms{<<}, \ms{>>}, and \ms{\~} on integers. Numeric values can be compared using the operators \ms{<}, \ms{<=}, \ms{>}, and \ms{>=}, and any values can be tested for equality using \ms{==}, \ms{!=}, \ms{is}, and \ms{is not}. Any object can be tested for truth value. The objects \ms{None}, \ms{False}, \ms{0}, \ms{0.0}, \ms{''}, \ms{[]}, and \ms{\string{\string}} are considered false; any other object is considered true.\\
    \indent OPythn inherits the control structures \ms{if}, \ms{for}, and \ms{while} from Python. Like in Python, consecutive lines at the same indentation level belong to the same block. Conditional expressions of the form \ms{<expr> if <condition> else <expr>} are also valid.

    \subsection{Functions}
    OPythn supports higher-order functions and nested functions. In OPythn, functions cannot modify variables outside their scope, unless one of the keywords \ms{global} or \ms{nonlocal} is used. OPythn supports anonymous functions using the keyword \ms{lambda}. OPythn functions take positional arguments only, and specifying default values is not allowed.\\
    \indent The following functions comprise OPythn's standard library and are included in a normal installation. Some of these functions are implemented in OPythn itself while others are implemented in OCaml. Their names are reserved and cannot be used in user-defined identifiers.
    \begin{center}
        \begin{tabular}{cccccc}
            \ms{abs()} & \ms{bin()} & \ms{bool()} & \ms{chr()} & \ms{divmod()} & \ms{enumerate()}\\
            \ms{filter()} & \ms{float()} & \ms{hash()} & \ms{hex()} & \ms{input()} & \ms{int()}\\
            \ms{isinstance()} & \ms{issubclass()} & \ms{iter()} & \ms{len()} & \ms{map()} & \ms{max()}\\
            \ms{min()} & \ms{next()} & \ms{oct()} & \ms{open()} & \ms{ord()} & \ms{pow()}\\
            \ms{print()} & \ms{range()} & \ms{repr()} & \ms{reversed()} & \ms{round()} & \ms{sorted()}\\
            \ms{str()} & \ms{sum()} & \ms{super()} & \ms{type()} &&
    \end{tabular}
    \end{center}

    \subsection{Classes and objects}
    OPythn is object-oriented and allows the user to define classes. However, nested classes and multiple inheritance are not supported. When creating a class, the following special methods can be added to indicate how the class should behave: \ms{\_\_init\_\_}, \ms{\_\_eq\_\_}, \ms{\_\_ne\_\_}, \ms{\_\_str\_\_}.\\
    OPythn does not allow private attributes and methods. In general, an object's attributes and methods can be accessed and mutated by functions outside the class' definition. All Python values are objects, but the user is not allowed to subclass built-in types.
    \subsection{Error handling}
    OPythn does not include support for exceptions. When an error occurs, the program terminates/returns to the REPL and a stack trace is printed.

    \subsection{Grammar}
    This is the complete OPythn grammar specification:
    \begin{lstlisting}[language=python]
    # Start symbols
    single_input: NEWLINE | simple_stmt | compound_stmt NEWLINE
    file_input: (NEWLINE | stmt)* ENDMARKER
    
    funcdef: 'def' NAME parameters ':' suite
    classdef: 'class' NAME ['(' [arglist] ')'] ':' suite
    parameters: '(' [arglist] ')'
    vararglist: arg (',' arg)*
    arg: NAME
    
    stmt: simple_stmt | compound_stmt
    simple_stmt: small_stmt (';' small_stmt)* [';'] NEWLINE
    small_stmt: (expr_stmt | del_stmt | flow_stmt | import_stmt
                 global_stmt | nonlocal_stmt)
    expr_stmt:  
    compound_stmt: if_stmt | while_stmt | for_stmt | funcdef |
        classdef
    global_stmt: 'global' NAME (',' NAME)*
    nonlocal_stmt: 'nonlocal' NAME (',' NAME)*
    
    # Control structures
    if_stmt: 'if' test ':' suite ('elif' test ':' suite)*
        ['else' ':' suite]
    while_stmt: 'while' test ':' suite ['else' ':' suite]
    for_stmt: 'for' exprlist 'in' testlist ':' suite
        ['else' ':' suite]
    
    # Tests and expressions
    test: or_test ['if' or_test 'else' test]
    or_test: and_test ('or' and_test)*
    and_test: not_test ('and' not_test)*
    not_test: 'not' not_test | comparison
    comparison: expr (comp_op expr)*
    comp_op: '<'|'>'|'=='|'>='|'<='|'!='|'in'|'not' 'in'|
        'is'|'is' 'not'
    expr: xor_expr ('|' xor_expr)*
    xor_expr: and_expr ('^' and_expr)*
    and_expr: shift_expr ('&' shift_expr)*
    shift_expr: arith_expr (('<<'|'>>') arith_expr)*
    arith_expr: term (('+'|'-') term)*
    term: factor (('*'|'/'|'%'|'//') factor)*
    factor: ('+'|'-'|'~') factor | power
    power: atom_expr ['**' factor]
    atom_expr: atom trailer*
    atom: (NAME | NUMBER | STRING+ | '...' | 'None' |
        'True' | 'False')
    trailer: '(' [arglist] ')' | '[' subscript ']' | . NAME
    subscriptlist: subscript (',' subscript)* [',']
    subscript: test | [test] ':' [test] [sliceop]
    sliceop: ':' [test]
    exprlist: expr (',' expr)* [',']
    testlist: test (',' test)* [',']
    dictorsetmaker: ( ((test ':' test | '**' expr)
                       ((',' (test ':' test)* [','])) |
                      (test (',' test)* [','])) )
    arglist: argument (',' argument)* [',']

    argument (test | test '=' test)
    \end{lstlisting}

% IMPLEMENTATION
\section{Implementation}
Both the OPythn bytecode compiler and interpreter will be implemented in OCaml. This section gives a rough overview of the process by which OPythn source code is handled and executed.
    \subsection{Input/Output}
    The OPythn program can be run directly in a Unix terminal via the command \texttt{opythn}. When run without arguments, this command launches a read-eval-print loop into which the user can enter commands. In interactive mode, the interpreter evaluates a valid command as soon as it is received:
    \begin{lstlisting}
    OPythn (Interactive), version 0.0.0
    :? for help, :q to quit
    ]=> print('Hello, world!')
    Hello, world!
    \end{lstlisting}
    Alternatively, OPythn source code can be defined in a separate \texttt{.opy} file and provided to the interpreter as a program argument. In this, mode, OPythn immediately compiles and interprets the source code directly.
    \subsection{Lexical Analysis}
    Upon reading source code, the OPythn front-end passes the data as a string to a series of functions that lex the code into tokens. The lexer will be created with the help of \texttt{ocamllex}, a program that generates a deterministic finite automaton in OCaml. This resulting lexer matches regular expressions in the string to convert chunks of characters into the correct tokens. 
    \subsection{Parsing}
    The tokens produced by the lexer is then fed into the parser, which will be implemented according to the grammar rules outlined in Section 2.3 and using \texttt{ocamlyacc}, an OCaml parser generator. The result will be an abstract syntax tree that represents the structure and semantics of the OPythn program. 
    \subsection{Bytecode}
    OPythn evaluates an abstract syntax tree by generating intermediary bytecode, which is evaluated by a virtual stack machine, as in many conventional Python implementations. As an example of what this looks like, consider this simple algorithm for integer multiplication in terms of addition:
    \begin{lstlisting}[language=python]
    x = int(input('Enter the multiplicand: '))
    y = int(input('Enter the multiplier: '))

    acc = 0
    while y > 0:
        if y % 2 == 0:
            x *= 2
            y //= 2
        else:
            acc += x
            y -= 1

    print('The product is: ', acc)
    \end{lstlisting}
    The disassembled output produced by the CPython interpreter looks like this:
    \begin{lstlisting}
      0 LOAD_NAME                0 (int)
      2 LOAD_NAME                1 (input)
      4 LOAD_CONST               0 ('Enter the multiplicand: ')
      6 CALL_FUNCTION            1
      8 CALL_FUNCTION            1
     10 STORE_NAME               2 (x)

     12 LOAD_NAME                0 (int)
     14 LOAD_NAME                1 (input)
     16 LOAD_CONST               1 ('Enter the multiplier: ')
     18 CALL_FUNCTION            1
     20 CALL_FUNCTION            1
     22 STORE_NAME               3 (y)

     24 LOAD_CONST               2 (0)
     26 STORE_NAME               4 (acc)

     28 SETUP_LOOP              58 (to 88)
     30 LOAD_NAME                3 (y)
     32 LOAD_CONST               2 (0)
     34 COMPARE_OP               4 (>)
     36 POP_JUMP_IF_FALSE       86

     38 LOAD_NAME                3 (y)
     40 LOAD_CONST               3 (2)
     42 BINARY_MODULO
     44 LOAD_CONST               2 (0)
     46 COMPARE_OP               2 (==)
     48 POP_JUMP_IF_FALSE       68

     50 LOAD_NAME                2 (x)
     52 LOAD_CONST               3 (2)
     54 INPLACE_MULTIPLY
     56 STORE_NAME               2 (x)

     58 LOAD_NAME                3 (y)
     60 LOAD_CONST               3 (2)
     62 INPLACE_FLOOR_DIVIDE
     64 STORE_NAME               3 (y)
     66 JUMP_ABSOLUTE           30

     68 LOAD_NAME                4 (acc)
     70 LOAD_NAME                2 (x)
     72 INPLACE_ADD
     74 STORE_NAME               4 (acc)

     76 LOAD_NAME                3 (y)
     78 LOAD_CONST               4 (1)
     80 INPLACE_SUBTRACT
     82 STORE_NAME               3 (y)
     84 JUMP_ABSOLUTE           30
     86 POP_BLOCK

     88 LOAD_NAME                5 (print)
     90 LOAD_CONST               5 ('The product is: ')
     92 LOAD_NAME                4 (acc)
     94 CALL_FUNCTION            2
     96 POP_TOP
     98 LOAD_CONST               6 (None)
    100 RETURN_VALUE
    \end{lstlisting}
    OPythn compiled source will look largely the same, but will not be converted into bitstrings. Instead, an \texttt{instr} datatype will be used to capture the same information as CPython's two-byte instructions. A different type tag will be used for each bytecode instruction.
    \begin{lstlisting}[language=caml]
        type instr =
          LOAD_NAME of int
        | LOAD_CONST of int
        (* ... *)
        | RETURN_VALUE
    \end{lstlisting}
    The advantage of this representation is that the interpreter can easily pattern-match on instruction tags. When expressed in this way, the above bytecode would be represented as the following \texttt{instr array}.
    \begin{lstlisting}[language=caml]
    [| LOAD_NAME 0;
       LOAD_NAME 1;
       LOAD_CONST 0;
       CALL_FUNCTION 1;
       CALL_FUNCTION 1;
       STORE_NAME 2;
       (* ... *)
       RETURN_VALUE |]
    \end{lstlisting}
    \subsection{Virtual Machine}
    Finally, the bytecode is interpreted by a stack-based virtual machine that produces the desired output. We can define a datatype specifically for stack items, so that any sort of data can be at the top of the stack. Then the stack can be represented as an \texttt{item list}, because we will only ever have to access the first few elements in the (linked) list. Many bytecode instructions access and manipulate the stack directly. For example, the instruction \texttt{POP\_TOP} removes the top element of the stack.\\
    \indent Additionally, the interpreter has access to an array of names and and array of constants, both of which were created upon code generation. For example, the instruction \texttt{LOAD\_NAME 1} accesses the name array at index 1 and pushes the associated object onto the stack; in our case above, this was the function \texttt{input}. Likewise, the instruction \texttt{LOAD\_CONST 5} accesses the constant array at the index 5 and pushes its value onto the stack.\\
    \indent Control-flow logic is implemented using the stack using certain instructions that tell the interpreter to jump to other parts of the bytecode. For example, at line 46 of the example bytecode, the top two elements on the stack are the values \texttt{0} and \texttt{y \% 2}. The \texttt{COMPARE\_OP 2} instruction checks if the top two elements of the stack are equal, and then pushes a boolean value onto the stack. The next instruction, \texttt{POP\_JUMP\_IF\_FALSE 68}, pops this boolean off the stack and jumps to line 68 if the boolean was false. Loops are implemented in a similar way. The instruction \texttt{SETUP\_LOOP} \textit{n} designates the next \textit{n} instructions as a block, and a test has to be run every loop to determine if the block should be exited via a \texttt{JUMP} instruction.\\
    \indent The evaluation procedure is performed by a recursive function \texttt{eval : int -> item list -> int}, in a context where the instruction, name, and constant arrays are defined. The \texttt{eval} function has as its inputs a line number and the current stack, and each time it is called, it performs the appropriate stack manipulations before calling itself again with the next line to evaluate. When \texttt{eval} runs out of instructions to evaluate (i.e. reaches the last instruction in the array with no \texttt{JUMP} command), the program halts and an exit code is returned.


\begin{thebibliography}{1}
    \bibitem{pythonref} Guido van Rossum. \textit{The Python Language Reference}. Python Software Foundation, 2019.
\end{thebibliography}

\end{document}
